%\section*{Future Research Work}
%\begin{itemize}
%\item Effect of heat generated due to the transient on performance of Current Transformer
%
%\item Effect of switching ON/OFF ckt isolation of circuit Breaker on Current Transformer
%
%\item Effect of super imposed voltage and current transient on source side and Load side
%\end{itemize}
%\clearpage

%-------------------------References--------------------------------------------------------
\bibliographystyle{IEEEtran-mod}
\addcontentsline{toc}{chapter}{\numberline{}References}
\bibliography{References}
\clearpage

%-----------------------PUBLICATIONS---------------------------------------------------------
\section*{Publications}
\addcontentsline{toc}{chapter}{\numberline{}Publications}
\begin{enumerate}
\item \textquotedblleft ~Partial ~Discharge ~Analysis ~in ~High ~Voltage ~Current Transformers\textquotedblright, International Refereed Journal of Engineering and Sciences (IRJES), ISSN (Online) 2319-183X, Print 2319-1821, Vol. 6, Issue 5, May 2017, pp. 15-23.

\item \textquotedblleft Partial Discharge with Short Circuit and Transients Generated on Current Transformer\textquotedblright, 5th ~International ~Conference ~on ~Engineering ~Technology, Science and Management Innovation (ICETSMI-2017) at (IETE) Institution of Electronics and Telecommunication Engineers, 30th April 2017, ISBN: 978-81-933746-7-2, pp. 200-207.
\end{enumerate}
\clearpage

%-----------------------Appendix-----------------------------------------------------

\chapter*{Appendix}
\addcontentsline{toc}{chapter}{\numberline{}Appendix}
\setlength{\parskip}{1em}
\section*{Isolation Voltage}
The initial design of a device with an insulation barrier includes a choice of materials and dimensions to achieve an isolation voltage rating.

The isolation rating is the voltage level under specified conditions that the device will withstand without breakdown.

\section*{Isolation Voltage Failure Modes}
Isolation breakdown in devices can occur in several ways.

In an insulator, electrons are tightly bound to atoms and molecules. With moderate gradient ~potential, ~some ~electrons ~are ~pulled ~free ~of ~their ~bonds ~to be ~later recaptured in collisions with neighboring atoms or molecules. As the gradient is increased beyond the Intrinsic Dielectric Strength of the material, collisions occur with sufficient strength to free more electrons than are captured, resulting in a disruptive breakdown called Intrinsic Strength Breakdown. In another type of breakdown, a path across the surface of a device can lead to carbonization and possibly even Surface Flashover.

Another more complex cause of degradation failure inside the volume of the insulation material is Erosion Breakdown which will be later presented.

In High Voltage Leakage testing, the tester applies line frequency AC voltage and monitors the resulting device current to not exceed a certain limit. A good device will have both resistive and capacitive leakage components that are proportional to the instantaneous applied voltage. The resistive component of device impedance is typically in the order of Gigohms. A Leakage failure occurs when this resistance is degraded. Excessive device capacitance could also cause leakage failure, though a physical mechanism to cause this seems improbable.

\section*{Partial Discharge}
Partial discharges are the source of Erosion Breakdown which affects the long term life of an insulator. Partial discharges are discharges that do not completely bridge the insulation between the terminals. These discharges are termed “partial” because they occur in areas that occupy a small portion of the electrical path length and are limited in magnitude because they are in series with mostly good insulation (which may eventually degrade). These discharges can occur in insulators that contain gaseous inclusions, cavities, or voids.

\section*{Electric Field Basics--Electric Flux Density "D"}
Gauss’s Theorem states that the Electric Flux Density at a distance $r_1$ from a concentrated charge Q on a small conductive sphere may be determined by enlarging the sphere until its radius is equal to $r_2$ (as the charge is distributed in equilibrium on the surface). The magnitude of the Electric Flux Density at $r_2$ is the charge Q divided by the surface area of the enlarged sphere.

\section*{Electric Field Intensity}
The strength and direction of the Electric Field Intensity of any point in an electric field may be measured by the force and direction upon a test charge. Electric Field Intensity measured in Newtons per Coulomb (dimensionally equivalent to Volts per meter) may be thought of as the force effectiveness of the electric field.

\section*{Medium Relative Permittivity (e\textsubscript{r})}
\begin{tabular}{l l}
Vacuum	& 1 \\
Air	& 1.006\\
Styrofoam	& 1.03\\
Polystyrene	& 2.7\\
Plexiglass	& 3.4\\
Amber	& 3\\
Rubber &3 \\
\end{tabular}

\section*{Rated voltage}
The rated voltage is the maximum voltage (phase-phase), expressed in kV rms, of the system for which the equipment is intended. It is also known as maximum system voltage.

\section*{Rated insulation level}
The combination of voltage values which characterize the insulation of an instrument transformer with regard to its capability to withstand dielectric stresses.
\section*{Lightning impulse test}
The lightning impulse test is performed with a standardized wave shape – 1.2/50 $\mu s$ – for simulation of lightning overvoltage.
\section*{Rated Power Frequency Withstand Voltage}
This test is to show that the apparatus can withstand the power frequency over-voltages that can occur.

The Rated Power Frequency Withstand voltage indicates the required withstand voltage. The value is expressed in kV rms.

\section*{Rated SIWL}
For voltages ≥300 kV the power-frequency voltage test is partly replaced by the switching impulse test. The wave shape 250/2500 μs simulates switching over-voltage. The rated Switching Impulse Withstand Level (SIWL) indicates the required withstand level phase-to-earth (phase to- ground), between phases and across open contacts. The value is expressed in kV as a peak value.
\section*{Rated Chopped Wave Impulse Withstand voltage, Phase-to-earth}
The rated chopped wave impulse withstand level at 2 $\mu s$ and 3 $\mu s$ respectively, indicates the required withstand level phase-to-earth (phase-to-ground).

\section*{Rated frequency}
The rated (power) frequency is the nominal frequency of the system expressed in Hz, which the instrument transformer is designed to operate in Standard frequencies are 50 Hz 

\section*{Ambient temperature}
Average 24 hours ambient temperature above the standardized +35 \textdegree C influences the thermal design of the transformers and must, therefore, be specified.

\section*{Creepage distance}
The creepage distance is defined as the shortest distance along the surface of an insulator between high voltage and ground. The rated currents are the values of primary and secondary currents on which performance is based. 

\section*{Rated primary current}
The rated current (sometimes referred to as rated current, nominal current or rated continuous current) is the maximum continuous current the equipment is allowed to carry. The current is expressed in an rms. The maximum continuous thermal current is based on average 24 h ambient temperature of +35 \textdegree C. It should be selected about 10 - 40\% higher than the estimated operating current. The closest standardized value should be chosen.

\section*{Extended current ratings}
A factor that multiplied by the rated current gives the maximum continuous load current and the limit for accuracy. Standard values of extended primary current are 120, 150 and 200\% of rated current. Unless otherwise specified, the rated continuous thermal current shall be the rated primary current.
\section*{Rated secondary current}
The standard values are 1, 2 and 5 A. 1 A gives an overall lower burden requirement through lower cable burden.
\section*{Rated short-time thermal current (I\textsubscript{th})}
The rated short-time withstand current is the maximum current (expressed in kA rms) which the equipment shall be able to carry for a specified time duration. Standard values for duration are 1 or 3 s. $I_th$ depends on the short-circuit power of the grid and can be calculated from the formula:
\[
I_{th} = Pk (MW) / Um (kV) \times \sqrt{3} kA.
\]
\section*{Rated dynamic current (I\textsubscript{dyn})}
The dynamic short-time current is according to IEC, $I_{dyn} = 2.5 \times I_{th}$ and according to IEEE, $I_{dyn} = 2.7 \times I_{th}$

\section*{Reconnection}
Current transformer can be designed with either primary or secondary reconnection or a combination of both to obtain more current ratios.
\section*{Primary reconnection}
The ampere-turns always remain the same and thereby the load capacity (burden) remains the same. The short-circuit capacity, however, may be reduced for the lower ratios. Primary reconnection is available for currents in relation 2:1 or 4:2:1.
\section*{Secondary reconnection}
Extra secondary terminals (taps) are taken out from the secondary winding. The load capacity drops as the ampere-turns decrease on the taps, but the short-circuit capacity remains constant. Each core can be individually reconnected.
\section*{Burden}
The external impedance in the secondary circuit in ohms at the specified power factor. It is usually expressed as the apparent power in VA, which is taken up at rated secondary current. It is important to determine the power consumption of connected meters and relays including the cables. Unnecessarily high burdens are often specified for modern equipment. Note that the accuracy for the measuring core, according to IEC, can be outside the class limit if the actual burden is below 25\% of the rated burden.

\section*{Accuracy}
The accuracy class for measuring cores is according to the IEC standard given as 0.2, 0.2S, 0.5, 0.5S or 1.0 depending on the application. For protection cores, the class is normally 5P or 10P. Other classes are quoted on request, e.g., class PR, PX, TPS, TPX or TPY.
\section*{Resistance}
The secondary winding resistance at 75 \textdegree C
\section*{Instrument Security Factor (IFS)}
To protect meters and instruments from being damaged by high currents, an FS factor of 5 or 10 is often specified for measuring cores. This means that the secondary current will increase maximum 5 or 10 times when the rated burden is connected. FS10 is normally sufficient for modern meters.
\section*{Accuracy Limit Factor (ALF)}
The protection cores must be able to reproduce the fault current without being saturated. The over current factor for protection cores is called ALF. ALF = 10 or 20 is commonly used. Both FS and ALF are valid at rated burden only. If lower burden the FS and ALF will increase.
\setlength{\parskip}{0em}
\clearpage

%-----------------------ACKNOWLEDGEMENT------------------------------------------------------

\begin{center}
\large \textbf{ACKNOWLEDGEMENT}
\addcontentsline{toc}{chapter}{\numberline{}Acknowledgement}
\end{center}

\setlength{\parskip}{1em}
\justify \normalsize I, take this opportunity to thank my guide \textbf{Dr. W. Z. Gandhare, Director, G. S. Moze College of Engineering, Pune} for his intense valuable guidance and constant motivation during the entire research work.

I wish to thank Internal Monitoring Committee Members \textbf{Dr. M. G. Shaikh, Dr. Mrs. A. S. Bhalchandra, Dr. R. V. Shetkar} and \textbf{Dr. A. G. Thosar}, Head of Electrical Engineering Department, Government College of Engineering, Aurangabad for their valuable support, suggestions and extending all the facilities for completion of this research work. I would like to appreciate the support and help given by my colleagues and all staff members of the institute.

I am thankful to \textbf{Dr. P. V. Murnal}, Principal, Government College of Engineering, Aurangabad for his valuable suggestions which helped me to accomplish this task. 

Finally, I would like to thank my brothers \textbf{Dr. Deepak, Mrs. Nilima} and \textbf{Dr. Vivek, Mrs. Jayeshsree} for their inspiration from childhood to till this study, wife \textbf{Mrs. Anupama}, children \textbf{Veena} and \textbf{Vaibhav} for their help and consistent encouragement without which it could not be possible for me to complete the research work.


\setlength{\parskip}{0em}
\vspace{2cm}
\begin{flushright}
\textbf{Vinayak Shankarrao Deolankar}
\end{flushright}