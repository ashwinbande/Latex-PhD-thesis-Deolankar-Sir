This research study consist of effect of Partial Discharge on high voltage oil filled current transformers. As high voltage electrical equipments are under various stresses such as thermal, mechanical, insulation strength, electrical stress which will deteriorate the performance and reliability of instrument. Also the defects during the manufacturing process, design aspects, ageing effect, voids formed in insulating material oil or solid plays major role in effective functioning of substation in totality.

In this study different sizes of voids in the insulating material are considered. Voids forms the capacitance in the electrical circuit and will have the different electrical effects in the circuit. Capacitance for different void sizes are tabulated.

For study and analysis purpose the mathematical model for oil filled current transformers of 145 kV up to 3000 A and void in oil insulation were considered. Charge due to various capacitance, the effect of diameter, volume were studied and analysed for the effect on partial discharge. 

A Simulation models for 145 kV current transformer current up to 3000 A were used in MATLAB to study the effect of various sizes of voids during steady state operations. The waveform pulses shows the effects in amplitude, width, number of pulses in sine wave. The short circuit and transient has high impact on life and deterioration of high voltage equipments. Since the voltage level and current are very high, electrical stress are generated, surrounding weak sections also contributes which leads to the formation of partial discharge and its impact leads to the failure of equipments. Simulation also done for different sizes of voids in insulation considering short circuit and transient conditions. The waveform show dense pulses, high amplitude and effect of transients on waveform. Laboratory results were studied on current transformers 145 kV, 3000/1, 2400/1, 600/5, 200/1 ratios for accuracy limit factory, ratio error, phase error, instrument security factor, high voltage test results during normal operating condition i.e steady state and effect after transients or deterioration of insulating parameters of materials. The ratio error and phase error results were compared for steady state and transient conditions. 
