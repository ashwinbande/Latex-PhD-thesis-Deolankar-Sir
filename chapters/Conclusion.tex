%Conclusion
\section{Conclusions}
\begin{itemize}
\item The Partial discharge inside the Current Transformer depends on the void present inside the oil and paper the magnitude and density of discharges is more during short circuit conditions 

\item The Partial discharge in current transformers increases with increase in magnitudes of voltage, deterioration of paper and oil insulation. The number of Partial Discharges, distribution, and magnitude of partial discharge, are sufficient to evaluate the performance of current transformers 

\item The magnitude, strength, and distribution of partial discharge pulses depend on the capacitance of voids. For same capacitance values in steady state condition and transient condition, partial discharge pulse will be different in magnitude and values 

\item Manufacturing Process needs to be carried out in controlled conditions with Low Vacuum Level less than o.5 mbar, Oil flow rate and temperature

\item The impact of PD in Steady State condition depends on relationship of Apparent Charge with Height, Volume, Radius of Voids

\item PD increases with magnitude and frequency of Short Circuit and Transients effect on Insulation
\end{itemize}
 
\pagebreak 

\section{Future Scope}
\begin{enumerate}
\item Effect of heat generated due to the transient on performance of Current Transformer 

\item Effect of switching ON/OFF circuit isolation of circuit Breaker on Current Transformer

\item Effect of super imposed voltage and current transient on source side and Load side
\end{enumerate}

\section{Applications}
From Simulink Model developed for study of Partial Discharge during Steady State, Short Circuit and Transient conditions Partial Discharge measurements can be done for Number of PD, Frequency of PD, Amplitude of PD for different Voltages up to 145 kV for various sizes of voids which will help in determining health and performance of High Voltage Current Transformers.

\clearpage
\section{Contributions}
\begin{itemize}
\item Analysis Model for High voltage oil filled current transformers 145 kV for comparison of waveforms in steady state and transients for different voids which give Amplitude, Frequency, Number of pulses of Partial Discharge

\item For different void sizes, partial discharge pulse was evolved for 145 kV Current Transformers during steady state and transients and are compared with to study effect on insulation strength of Oil and Papers

\item The results of performance distortion of oil filled current transformers 145 kV up to 3000 A by computational method and analytical method shows same relevance in steady state and transient conditions due to the voids in insulating material or occurrence of partial discharge
\end{itemize}
\clearpage