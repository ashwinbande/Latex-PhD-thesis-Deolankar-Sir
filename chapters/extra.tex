Dielectric breakdown		&\multirow{4}{*}{9.9}	& Breakdown to earth: 5\%							\\ 
													&& Internal breakdown across open pole, during opening operation = does 	not break the current: 1.9\% \\
													&& Other across open pole: 1.8\%						\\
													&& Breakdown between poles: 1.2\%					\\ \hline
													
													
\subsubsection{Chapter 1} Introduction - This chapter addresses the aspects containing introductory part. The necessity and the objectives of this research work are clearly mentioned. The overall theme of the complete research work is presented.
\subsubsection{Chapter 2} Literature Survey - A summary of the exhaustive literature survey is represented. An overview of arc modeling, Transient Recovery Voltage, Dynamic Contact Resistance Measurement (DCRM), Standards related to circuit breaker timings are discussed in detail.
\subsubsection{Chapter 3} System Development - Computational model of IEEE network for TRV study under different fault conditions is developed in EMTP-RV. Similarly analytical and mathematical models are developed. Dynamic contact resistance measurement is explained. Mathematical treatment is explained in detail with relevant references.
\subsubsection{Chapter 4} Performance Analysis – Results of analytical and computational methods are presented. Justification for difference is given. Measured and collected data of DCRM from field is analyzed in detail using HISAC ULTIMA test manager software and new algorithm is proposed to detect the contact anomaly. Computer program in Java is developed to determine the health of circuit breaker.
\subsubsection{Chapter 5} Conclusions - In this chapter conclusions of the research work, future work and  applications are presented.


\begin{table}[!htbp]
\begin{threeparttable}
\renewcommand{\arraystretch}{1.3}
\caption{Percentage of Maf Rate and Mif Rate per Failure Mode, Third Inquiry.}
\label{table:Percentage of Maf Rate}
\centering
\small

\begin{tabular}{| l | c | l | l | l | l |} \hline
\multicolumn{2}{|c|}{1} & 2 & 3 & 4 & 5 \\ \hline
\multirow{2}{*}{1}		& 2 & 3 & 4 & 5 & 6 \\ \cline{2-6}
				 		& 2 & 3 & 4 & 5 & 6 \\ \hline
				 		
\multirow{4}{*}{1}		& 2 & 3 & 4 & 5 & 6 \\ \cline{2-6}
						& 2 & 3 & 4 & 5 & 6 \\ \cline{2-6}
						& 2 & 3 & 4 & 5 & 6 \\ \cline{2-6}
				 		& 2 & 3 & 4 & 5 & 6 \\ \hline
\end{tabular}
\end{threeparttable}
\end{table}

%----------------------SUBFIGURE WITH FOOTNOTE CAPTIONS-------------------------------------
    \begin{subfigure}[b]{0.3\textwidth}
        \centering
        \begin{minipage}{\textwidth}
		\centering
		\begin{flushleft}
		\footnotesize a. Inside porcelain wall of stationary main contact is seen with white gray powder. Arcing led to decomposition of SF6 gas forming metal fluorides and metal sulphates.\\
b. Stationary arc contact is seen with slight burning marks at the tip indicating arcing for substantial times and duration
		\end{flushleft}
		\end{minipage}
    \end{subfigure}

%------------------------------BOXED EQUATIONS----------------------------------------------
    \boxed{\frac{f^2}{1-f^2}P = 3.16 \times 10^{-7} T^{5/2} e^{-e V_i/ kT}}
%------------------------------TEXT SIZES---------------------------------------------------
6  \tiny
8  \scriptsize
10 \footnotesize
12 \small
14 \normalsize
16 \large
18 \Large
20 \LARGE
22 \huge
24 \Huge



%-------------------------------WRAP FIGURE--------------------------------------------------
\usepackage{wrapfig}
You can use intextsep parameter to control additional space above and below the figure: 
\setlength\intextsep{0pt}

\begin{wrapfigure}{r}{0.25\textwidth} %this figure will be at the right
    \centering
    \includegraphics[width=0.25\textwidth]{mesh}
\end{wrapfigure}
 
There are several ways to plot a function of two variables, 
depending on the information you are interested in. For 
instance, if you want to see the mesh of a function so it 
easier to see the derivative you can use a plot like the 
one on the left.
 
 
\begin{wrapfigure}{l}{0.25\textwidth}
    \centering
    \includegraphics[width=0.25\textwidth]{contour}
\end{wrapfigure}
 
On the other side, if you are only interested on
certain values you can use the contour plot, you 
can use the contour plot, you can use the contour 
plot, you can use the contour plot, you can use 
the contour plot, you can use the contour plot, 
you can use the contour plot, like the one on the left.

%------------------------------SIZE DEFINITIONS---------------------------------------------
Abbreviation	Definition
pt				A point, is the default length unit. About 0.3515mm
mm				a millimetre
cm				a centimetre
in				an inch
ex				the height of an 'x' in the current font
em				the width of an 'm' in the current font
\columnsep		distance between columns
\columnwidth	width of the column
\linewidth		width of the line in the current environment
\paperwidth		width of the page
\paperheight	height of the page
\textwidth		width of the text
\textheight		height of the text
\unitleght		units of length in the picture environment.
